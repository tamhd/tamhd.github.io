\documentclass[a4paper, 12pt]{scrartcl}
\usepackage{amsmath,amsfonts}
\usepackage{scrpage2}
\usepackage{scrtime}
\usepackage{setspace}
\usepackage{natbib}
\usepackage{enumerate}
\usepackage{graphicx}
\usepackage{verbatim}
%\usepackage{harvard}
\usepackage{color}
\usepackage{url}
\usepackage[margin=1in]{geometry}

\pagestyle{scrheadings}
\setlength\headheight{25.2pt}
\ohead{\thepage}
\chead{DUC TAM, HOANG}
\ihead{Letter of Motivation}
\setheadsepline[head]{3pt}
\renewcommand{\headfont}{\normalfont}
\newcommand{\sqz}{\hfill\phantom{.}}
\newcommand{\thickhline}{\noalign{\hrule height 0.8pt}}
\renewcommand{\baselinestretch}{1.1} 
\setlength\parindent{0mm}
\setlength\parskip{5mm}

\begin{document}
%\doublespacing
%\section{Research}

If we consider a model in Machine Translation as an \emph{isolated system} which can change sentence \emph{form} from state \emph{e} to state \emph{f}, 
the sentence meaning has to be preserved during the conversion. 
Since the dawn of Statistical Machine Translation (SMT), SMT enthusiasts have come a long way but yet to come up with a ``perfect'' model which supports meaning preservation.
My PhD dissertation will investigate a potential adequate model which conserves 
the input meaning. 
One might say that it is an impossible task, due to the complex meaning that an expression might convey. 
A simplier goal to minimize the
loss of meaning in translation is both challenging and interesting. 


To define the meaning of a unit (a word, a phrase, etc), the model has to consider the interaction between the unit itself and all other elements that come to contact. 
In most circumstances, this is done by introducing new factors to the model.
Then, there is a new task answering two questions: ``what is the factor of meaning?'' and ``how to integrate the new factor?''.

%PLAN
I plan to finish my Ph.D in a total of four years. 
The first 6 months is scheduled to examine an old factor called ``domain''. 
It will be built based on my old project, which targets \emph{domain adaptation}. 
In some situations, ``domain'' has a clear definition. In other situations, it is not easy to define this factor. 
To the best of my knowledge, there have been only a few studies of \emph{domain} effect on translation quality and usage of this factor in the model.
Subsequently, as the premier of my research is to discover a good model,
I plan to explore the potential of new factors until the model serves its purpose.
It is possible that several tasks (progress reports, conference papers, test data creation, etc) will be performed in parallel. 
Finally, I wish to spend the last 9 months writing my dissertation.

%WHY ILLC
The discerning, independent, creative, innovative and international ethos of ILLC will be the perfect environment for me to do research.
Resources at ILLC will enable me to test all of my hypotheses. 
Moreover, world-class scientists at ILLC will be a great mentor to guide me on the right trajectory.
Especially, ILLC has an excellent reputation among SMT comunity.
My PhD will broaden my horizon to answer the question ``What is the best model for machine translation'', ranging from a plain phrase-based statistical model to a possibility of syntax-based statistical model. 

%WHY me
Over the past years, I have participated in three different SMT-related projects at Institute of Formal and Applied Linguistics, Charles University. They include both teaching-oriented projects and research projects. 
Having a strong will to continue working in this area, I would like to apply for a PhD position at ILLC with specialization in \emph{Statistical Machine Translation}, group \emph{Language and Computation}. 
At the moment, I have not (yet) held a master degree but I will graduate this September. With an excellent GPA and a solid background in Computational Linguistics, I strongly believe that I am a qualified candidate.

\end{document}

